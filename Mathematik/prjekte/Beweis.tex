\documentclass[12pt, letterpaper]{article}
\usepackage{amsfonts}
\usepackage{amsmath}


\begin{document}
\section{Lemma1}
Theorem: Es gibt kein Element, nach dem alle nachfolgenden Elemente gleich sind.
\\\\
Sei es gibt ein solches Element $a_{x}$.
\\
Demnach: $\forall k \in \mathbb{N}(a_{x+k}=\varepsilon$)
\\\\
$a_{x+1} = \varepsilon$\\
\\Demnach: \[  
  (x+1)! = l\cdot 10^{k+1} + 10^{k}\cdot \varepsilon\\
\]
\[  
  
  (x+2)! = (x+1)\cdot l\cdot 10^{k}+(x+1)\cdot 10^{k}\cdot \varepsilon\\
  (x+1)\cdot l\cdot 10^{k}= l_{2}\cdot 10^{m+1}\\
  (x+1)\cdot 10^{k}\cdot \varepsilon = 10^{m}\cdot \varepsilon\\

\]


\end{document}
