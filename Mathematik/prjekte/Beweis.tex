\documentclass[12pt, letterpaper]{article}
\usepackage{hyperref}
\usepackage[utf8]{inputenc}
\usepackage{amsfonts}
\usepackage{amsmath}
\usepackage{lwarp-titlesec}
\usepackage{graphicx}
\graphicspath{\media\Aaron\USBStick}
\hypersetup{
    colorlinks=true,
    linkcolor=blue,
    filecolor=magenta,      
    urlcolor=cyan,
    pdftitle={Overleaf Example},
    pdfpagemode=FullScreen,
    }

\title{Mathewettbewerb 2025}
\author{Aaron Tsamaltoupis}

   \newtheorem{proof}{Proof}[section]
   \newtheorem{case}{Fall}[Proof]
   \newtheorem{theorem}{Theorem}[section]
   \newtheorem{corollary}{Corollary}[theorem]
   \newtheorem{lemma}[theorem]{Lemma}

\begin{document}
\maketitle
\tableofcontents
\newpage
\newpage
\section{Nr. 2}
\label{sec:Nr. 2}


\subsection{Theorem 1}
\label{sec:Theorem 1}
\textit{Für alle ganzen Zahlen n größer gleich 2 gilt, dass die letzte von 0 verschiedene Ziffer von n! eine gerade Ziffer ist.} \\\\
Sei ein beliebiges $a\in \mathbb{Z}^{+}$ und sei die letzte von 0 verschiedene Ziffer von $a!$ gerade.\\
Die Zahl a kann aufgeteilt werden in den Teil ab der letzten von 0 verschiedene Ziffer $2f$ und dem Rest $l$. Da diese letzte Ziffer gerade ist und nach ihr nur noch Nullen kommen können, muss $a!$ diese Form haben:\\
$a! = 2l\cdot 10^{k+1}+2f$, wobei $l\in \mathbb{N}\land f \in \mathbb{Z}_{+}\land f>5$\\\




zu beweisen ist also: \[  
  \forall n \in \mathbb{N}((n>1)\implies \exists l\in \mathbb{N}, f\in \mathbb{Z}^{+}(n! = 2l\cdot 10^{k+1}+2f\cdot 10^{k}, wobei\ \land f<5  ))\\
\]

Sei der Einfachkeit halber \[  
  P(n)\ \textnormal{iff}\ \exists l\in\mathbb{N}, f\in\mathbb{Z}^{+}(n! = 2l\cdot 10^{k+1}+2f\cdot 10^{k}, wobei\ \land f<5  ))\\

\]


\textbf{Beweis Theorem 1.1} \\
Sei n eine beliebige ganze Zahl.\\
n ist entweder ein Vielfaches von 5 oder nicht.
 \\Theorem 1.1 wird für diese beiden Fälle seperat bewiesen.\\

\\\newpage
\textbf{Theorem 1.1, Fall 1} \\
\label{sec:Fall 1}
n ist ein Vielfaches von 5.\\
$\exists m \in \mathbb{N}(n! = 5\cdot m)$\\

Es soll bewiesen werden, dass jedes $n! = (5m)!$ in der Form $(5m)! =2^{l}\cdot 5^{k}\cdot r $ aufgeschieben werden kann, wobei r weder durch 5 noch durch 2 teilbar ist und l größer als k ist.\\
$\forall m \in \mathbb{Z}^{+}(\exists l, k, r \in \mathbb{N}((5m)! = 2^{l}\cdot 5^{k}\cdot r \land $2\hspace{-0.4em}\not\hspace{0.025em}|\,r$
\land $5\hspace{-0.4em}\not\hspace{0.025em}|\,r$\land l>k
))$
\\\\
Es soll mittels vollständiger Induktion bewiesen werden.\\
Induktionshypothese:\\ sei 
$\forall m <m_{0}(\exists l, k, r \in \mathbb{N}((5m)! = 2^{l}\cdot 5^{k}\cdot r \land $2\hspace{-0.4em}\not\hspace{0.025em}|\,r$
\land $5\hspace{-0.4em}\not\hspace{0.025em}|\,r$\land l>k
))$

\\
\newpage
\textbf{Theorem1.1, Fall 2} \\

n ist kein Vielfaches von 5.\\
$\neg \exists m \in \mathbb{N}(n! = 5\cdot  m)$\\
Es soll mittels Induktion bewiesen werden, dass für alle $n\in \mathbb{N} \land n>1$ gilt, dass wenn n kein Vielfaches von 5 ist, $P(n))$  wahr ist.\\\\\textbf{Base case:} \\
$\neg \exists m \in \mathbb{N}(2! = 5m)$\\
$\implies 2! \neq 5m$\\
$2! = 2 =0\cdot 10^{1} +2\cdot 1\cdot 10^{0}$\\
$\implies P(2)$\\
\\\textbf{Induction step}\\
sei $(\neg \exists m\in \mathbb{N}(n! = 5m) \implies P(n))$(Induktionshypothese)\\

P(n) kann hier trotzdem in für alle Fälle als gegeben hingenommen werden, da selbst wenn $n! = 5m$ aus dem vorherigen Beweis für Theorem1.1, Fall1 hervorgeht, dass P(n) wahr ist.\\

P(n) ist hier trotzdem wahr für jedes beliebige $n>1$.
Sollte n kein Vielfaches von 5 sein, garantiert dies die Induktionshypothese. Ist n kein Vielfaches von 5 garantiert das der Beweis in \hyperref[sec:Fall 1]{Fall 1}

zu beweisen: $(n+1)\neq 5m \implies P(n+1)$\\

sei $\neg \exists m \in \mathbb{N}(5m =n+1)$\\

$n! = l\cdot 10^{k+1}+2f\cdot 10^{k}, f<5$ (aufgrund der Induktionshypothese, bzw Fall 1)\\\\
$(n+1)! = (l\cdot 10^{k+1}+2f\cdot 10^{k})\cdot (n+1)$\\
$= (n+1)\cdot l\cdot 10^{k+1}+2f(n+1)\cdot 10^{k}$\\\\
$(n+1)$ und $f$ sind beide keine Vielfache von 5, nach \hyperref[sec:Lemma1]{Lemma1} kann daher auch $2(n+1)\cdot f$ kein Vielfaches von 5 oder 10 sein.\\
$\exists m\in\mathbb{N}, f_{0}\in\mathbb{Z}^{+}(f_{0}<5\land (n+1) = 10m + 2f_{0})
$\\
$(n+1)! = l(n+1)\cdot 10^{k+1}+(10m+2f_{0})\cdot 10^{k}\\
(n+1)!=10^{k+1}((n+1)\cdot l+m)+10^{k}\cdot 2f_{0}\\
(n+1)!=10^{k+1}l_{0}+10^{k}\cdot 2f_{0}$, wobei $f_{0} < 5$
\\
$\implies P(n+1)$






\\\\







Es ist damit bewiesen, dass\\ $(\neg \exists m\in \mathbb{N}(n! = 5m) \implies P(n)) \implies(\neg \exists m\in\mathbb{N}((n+1) = 5m) \implies P(n+1))$\\

Dadurch ist für alle $n>2$ bewiesen, dass $\neg \exists m\in\mathbb{N}(n! = 5m)\implies P(n)$ 



\\\\




\newpage
\section{Lemmas}
\label{sec:Lemmas}

\subsection{Lemma1}
\label{sec:Lemma1}

\end{document}
