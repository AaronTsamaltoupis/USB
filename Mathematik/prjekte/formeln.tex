\documentclass[12pt, letterpaper]{article}
\documentclass[12pt, letterpaper]{article}
\usepackage{hyperref}
\usepackage[utf8]{inputenc}
\usepackage{amsfonts}
\usepackage{amsmath}
\usepackage{graphicx}

\begin{document}
\textbf{Lemma1: Alle Elemente der folge sind gerade Ziffern.}\\

Sei die letzte von 0 verschiedene Ziffer von x! eine gerade Zahl.
\\Es soll bewiesen werden, dass dann für (x+1)! dasselbe gilt.\\
\[  
  x! = l\cdot 10^{k+1} +2\cdot f\cdot  10^{k}, f<5
\]
\[  
   (x+1)!\\
  = (x+1)\cdot  (l\cdot 10^{k+1} +2\cdot f\cdot  10^{k})

\]

\[  
  = `
\]

\newpage
\section{Die Aufgabe}
\label{sec:Die Aufgabe}
sei eine Folge a definiert auf Folgende weise: \\
Das nte Element der folge a ist die letzte von 0 verschiedene ziffer der zahl n!.



\section{Begriffsklärung}
sei eine zahl x. Sei P(x) die letzte von 0 verschiedene ziffer dieser zahl.\\


\section{erarbeitete formeln}
\label{sec:erarbeitete formeln}
\begin{itemize}
	\item Sei in der folge a kommen nur die Zahlen 2, 4, 8, 6 vor.
	\item Sei es gibt nur endlich viele zweien in der Folge
	\item Es gibt also ein element $a_{x_{0}}$ der Folge a, welches die letzte zwei in der folge ist.\\
	\item \textbf{Formel:}	\[	
		x! = 10^{k} \cdot  l + 2\cdot  10^{k-1}
	\]
	\item \textbf{Formel:}	\[	
		(x+k)! = 10^{k}(l\cdot (x+1)(x+2)....(x+k))+2\cdot (x+1)(x+2).....(x+k)\cdot 10^{k-1})
	\]
	\item \textsc{goal:}	\[	
			2\cdot \frac{(x+k)!}{x!}\cdot  10^{k-1} = (l\cdot 10 +2	)10^{k-1}
	\]

\end{itemize}

\end{document}
