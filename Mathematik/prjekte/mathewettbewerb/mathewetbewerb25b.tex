\documentclass[12pt, letterpaper]{article}
\usepackage{hyperref}
\usepackage[utf8]{inputenc}
\usepackage{amsfonts}
\usepackage{amsmath}
\usepackage{amsthm}
\usepackage{lwarp-titlesec}
\usepackage{graphicx}
\usepackage{amssymb}
\graphicspath{\media\Aaron\USBStick}
\hypesetup{
    colorlinks=true,
    linkcolor=blue,
    filecolor=magenta,      
    urlcolor=cyan,
    pdftitle={Overleaf Example},
    pdfpagemode=FullScreen,
    }

\title{Mathematikwettbewerb 2025}
\author{Aaron Tsamaltoupis}

   \newtheorem{theorem}{Theorem}[section]
   \newtheorem{corollary}{Corollary}[theorem]
   \newtheorem{lemma}[theorem]{Lemma}

\begin{document}
\maketitle
\tableofcontents
\newpage
\newpage
\section{Nr 1}
\label{sec:Nr 1}

\newpage
\section{Nr 2}
\label{sec:Nr 2}
Für jede ganze Zahl $n\ge 2$ betrachten wir in der Dezimaldarstellung von $n!$ die letzte von Null verschiedene Ziffer.\\
Bestimme alle Ziffern, die mindestens einmal in dieser Folge vorkommen, und zeige, dass jede dieser Ziffern sogar unendlich oft vorkommt.\\\\

\end{center}
Sei eine funktion $f: \mathbb{Z^{+}}\rightarrow [1,2,3,4,5,6,7,8,9]$ definiert, sodass $f(n)$ die letzte von Null verschiedene Ziffer von n ist.
\begin{center}
  
$f(n) = \varepsilon$ \textit{iff}\varepsilon \in \{1,2,3,4,5,6,7,8,9\} \land   $\exists k, l(n = 10^{k}\cdot (10\cdot l+\varepsilon))$\\
\end{center}
Sei B die Menge aller Folgen b, die folgendermaßen definiert werden können: 
\begin{center}
  $b_{1} = k$ und für alle $n\in \mathbb{Z_{+}}$ gilt $b_{n} = \frac{(k+n-1)!}{(k-1)!} $
\end{center}
Alle Folgen $b\in B$ haben also folgende Elemente:
\begin{center}
  $b_{1} = k\\
  b_{2} = k\cdot (k+1)\\
  b_{3} = k\cdot (k+1) \cdot  (k+2\\
  b_{4} = ...

\end{center}
Sei eine weitere Menge F, die Folgen beinhaltet, folgerndermaßen definiert:
\begin{center}
  $F = \{Fb:\exists b\in B(Fb_{n} = f(b_{n}))\}$
\end{center}
Das nte Element jeder Folge $Fb_{0}$ in F ist also die letzte von Null verschiedene Ziffer des nten Elementes einer Folge $b_{0}\in B$\\

Sei die Folge aus der Aufgabenstellung, die mit (2,6,4,2,2,..) beginnt die Folge $Fa = (2,6,4,2,2,...)$
\\Diese Folge kann folgendermaßen beschrieben werden:\\
$Fa_{n} = f(a_{n})$, wobei $a_{n}$ das nte Element einer weiteren Folge a ist, wobei a = (2!, 3!, 4!, ...)




Auch diese Folge a ist ein Element der Menge B:
\begin{center}
  \begin{math}
    a_{1}=2\\
    a_{2} = 2\cdot 3\\
    a_{3} = 2\cdot 3\cdot 4\\
    ...
  \end{math}
\end{center}
Da a also ein Element von B ist und $Fa_{n} = f(a_{n})$ ist Fa auch ein Element von F.


\begin{lemma}
  Es soll bewiesen werden, dass für alle Folgen $b\in B$ gilt, dass sobald es ein Element $b_{h_{0}}$ dieser Folge gibt, für das gilt \begin{center}
  $\exists p_{0},q_{0},m_{0}\in \mathbb{Z+}(2\nmid m_{0}\land 5\nmid m_{0} \land b_{h_{0}} = 10^{p_{0}}\cdot 2^{q_{0}}\cdot m_{0} )$
\end{center} 
,dann gilt für alle folgenden Elemente h dieser Folge b, dass es auch für sie ein $p, q$ und $m \in \mathbb{Z^{+}}$ gibt, wobei $(2\nmid m \land  5\nmid m \land h = 10^{p}\cdot 2^{q}\cdot m)$
\\\\
Sei

\end{lemma}
\begin{proof}
  
\end{proof}
\begin{lemma}
 Es soll nun bewiesen werden, dass es für jede Folge $b\in B$ein solches Element $b_{h_{0}}$ gibt. 
\end{lemma}
\begin{proof}
  
\end{proof}
\begin{lemma}
  Es soll bewiesen werden, dass wenn ein $n\in \mathbb{Z^{+}}$ in der Form $n= 10^{p}\cdot 2^{q}\cdot m$ notiert werden kann, wobei m weder durch 2 noch durch 5 teilbar ist, $f(n)$ eine gerade Ziffer ist.\\

  Zu beweisen:
\begin{center}
\begin{math}
  \forall n\in\mathbb{Z^{+}}(\exists(p,q,m)(2\nmid m\land 5\nmid m \land n = 10^{p}\cdot 2^{q}\cdot m)\implies f(n)\in [2,4,6,8])
\end{math} 
\end{center}
\end{lemma}
\begin{proof}
  
\end{proof}


Nach Lemma 2.1 gilt also, dass es ein Element $a_{h_{0}}$ der Folge a gibt, wonach alle nachfolgenden Elemente $a_{h}$ der Folge a in der Form $10^{q}\cdot 2^{p}\cdot m$ geschrieben werden können, wobei m weder durch 5 oder durch 2 teilbar sind.\\
Direkt das erste Element von a $a_{1} = 2!$ kann in dieser Form geschrieben werden: $2 = 10^{0}\cdot 2^{1}\cdot 1$, demnach können alle Elemente von a in dieser Form geschreiben werden.\\\\
Nach Lemma 2.3 gilt dann, dass $f(a_{k})$ für alle Elemente $a_{k}$ von $a$ eine gerade Ziffer ist.\\
Die Ausgangsfolge der Aufgabenstellung $Fa= (2,6,4,2,2,...)$ besteht also nur aus geraden Ziffern.\\\textbf{Keine andere Zahl außer 2,4,6, oder 8 kommt also in der Ausgangsfolge $Fa$ vor.} 
\begin{lemma}
  Es soll bewiesen werden, dass in jeder Folge $Fb \in F$ immer mindestens zwei verschiedene Elemente undendlich oft vorkommen.
\end{lemma}
\begin{proof}
  Es soll durch Widerspruch bewiesen werden. Sei also ein Element $Fb_{x_{0}}$ von Fb, ab dem alle folgenden Elemente $Fb_{x}$ nur noch den Wert $\varepsilon_{0}$ haben.\\
  $Fb_{x_{0}} = f(b_{x_{0}})$, wobei $b_{1} = k$\\
  Es gibt ein $b_{h+1}$ , sodass $h>x_{0}$ und $h+k$ kein Vielfaches von 5 ist. \\\\
    \begin{math}
      b_{h+1} =  \frac{(k+h)!}{(k-1)!} \\\\
      =\frac{(k+h-1)!}{(k-1)!}\cdot (k+h) \\\\
      =b_{h}\cdot (k+h)
    \end{math}\\\\
  $b_{h}$ muss von der Form $b_{h} = 10^{n}\cdot (10l+\varepsilon_{0})$ sein, da ansonsten $f(b_{h}) \neq \varepsilon_{0}$ und demnach $Fb_{h} \neq \varepsilon_{0}$\\\\
  \begin{math}
    \implies b_{h+1} = 10^{n_{1}}\cdot (10l_{1}+\varepsilon_{0}) \cdot (k+h)\\\\
    \implies 
  \end{math}

  \end{proof}
  \begin{lemma}
    \begin{math}
      \forall n_{1},n_{2}\in\mathbb{N}(f(n_{1}) = \varepsilon_{1} \land f(n_{2}) = \varepsilon_{2} \implies f(n_{1}\cdot n_{2})= f(\varepsilon_{1} \cdot \varepsilon_{2}))
    \end{math}
  \end{lemma}
  \begin{proof}
    
  \end{proof}
  \begin{lemma}
    Es soll bewiesen werden, dass jede Folge $b\in B$ mindestens ein Element $b_{n}$ hat, sodass $f(b_{n}) = 6$. Es soll also bewiesen werden, dass in jeder Folge $Fb \in F$ mindestens einmal die Zahl 6 vorkommt. 
  \end{lemma}

  \begin{proof}
    Es soll per Widerspruch bewiesen werden.\\
    Sei es gibt eine folge $b^{k}\inB$, sodass $\neg \exists n\in\mathbb{N}(f(b^{k}_{n}) =6)$.\\
    Sei ein element $b^{k}_{k_{1}}$ der Folge b.\\
    $f(b^{k}_{k_{1}}) \in \{2,4,8\}$\\
    Die folge $Fb^{k_{1}+1}=(f(k_{1}+1),f((k_{1}+1)\cdot (k_{1}+ 2)),f((k_{1}+1)\cdot (k_{1}+ 2)\cdot (k_{1}+3),...)$ enthält nach Lemma 2.4 mindestens zwei Elemente der Menge $\{2,4,6,8\}$ unendlich oft, daher enthält sie mindestens eine 2er Potenz (2,4,8) unendlich oft. 



  \end{proof}
  \end{document} 
