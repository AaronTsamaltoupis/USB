\documentclass[12pt, letterpaper]{article}
\usepackage{hyperref}
\usepackage{graphicx}
\usepackage{amsfonts}
\usepackage{amsmath}
\graphicspath{\media\Aaron\USBStick}
\usepackage[utf8]{inputenc}
\usepackage{lwarp-titlesec}
\hypersetup{
    colorlinks=true,
    linkcolor=blue,
    filecolor=magenta,      
    urlcolor=cyan,
    pdftitle={Overleaf Example},
    pdfpagemode=FullScreen,
    }

\title{Mathematik}
\author{Aaron Tsamaltoupis}


\begin{document}
\maketitle
	\begin{center}
\includegraphics[width = 10cm]{graphics/ghibli wallpaper all characters.jpg}
\end{center}
\newpage
\tableofcontents
\newpage
\section{Logic}
\label{sec:logic}
\subsection{akkentential logic}
\label{sec:sentential logic}
\subsubsection{Sentence symbols}
\label{sec:Sentence symbols}

\begin{tabular}{c|c|c|c } 
 \hline
  &symbol & verbose name& remarks \\ 
  \hline
	1 &( & left parenthesis &  \\ 
	2 &) & right parenthesis &\\ 
	 3 &$\neg$&negation symbol& not\\
	 4&$\land$&conjunction symbol& and\\
	 5&$\lor$&disjunction symbol& or\\
	 6&$\rightarrow $&conditional symbol&if, then\\
	 7&$\leftrightarrow $&biconditional symbol & if and only if (iff)\\
	 ...&A_{1}& first sentence symbol& \\
	  &A_{2}& second sentence symbol&\\
	  &...&&\\
	  &A_{n}&nth sentence symbol&\\
	  &...&&\\

 \hline
\end{tabular}
\begin{itemize}
	\item The first seven symbols are called the \textit{logical symbols}	Their translation is fixed.\\
	
		The logical symbols without the parenthesis are the \textit{sentential connective symbols}.	\\
		The sentence symbols $A_{i}$ are the \textit{parameters (non logical symbols)}. Their translation is not fixed.
	\item No sentence symbol is itself a sequence of other sentence symbols.\\
		That means:
		

\end{itemize}
\newpage
\subsubsection{expressions}
\label{sec:expressions}
An expression is a finite sequence of symbols.\\
bspw: $(\neg A_{1})$\\
If $\alpha$ and $\beta$ are expressions $\alpha\beta$ is the sequence of consisting of all the senctence symbols in $\alpha$ followed by all in $\beta$.\\
Sei\\ $\alpha$ = (\neg A_{1})\\
$\beta$=A_{2}\\

\\ $\alpha \rightarrow \beta$ then is  $(\neg A_{1}) \rightarrow A_{2}$ 

\\
If parameters are combined with the logical symbols expressions can be used to translate english sentences into the logical language.

\\\\Some expressions are nonsense. Bspw: $((\rightarrow A_{3}$

This is why \textit{grammatically correct}	expressions are defined as \hyperref[sec:well formed formulas]{well formed formulas}
\subsubsection{well formed formulas}

\label{sec:well formed formulas}
The definition of the well formed formulas aims to exclude the nonsensical expressions:\\
\begin{itemize}
	\item Every sentece symbol is a wff.
	\item If $\alpha$ and $\beta$ are wffs, then so are $\neg \alpha, (\alpha \land \beta)$, $(\alpha \lor \beta)$, $(\alpha \rightarrow \beta)$, 
\end{itemize}
\textbf{the formula-building operations}
\begin{enumerate}
	
	\item $\varepsilon_{\neg}(\alpha) = (\neg \alpha)$
	\item $\varepsilon_{\land }(\alpha, \beta) = (\alpha \land \beta)$
	\item $\varepsilon_{\lor }(\alpha, \beta)= (\alpha \lor \beta)$
	\item $\varepsilon_{\rightarrow }(\alpha, \beta) = (\alpha \rightarrow  \beta)$
	\item $\varepsilon_{\leftrightarrow }(\alpha, \beta) = (\alpha \leftrightarrow  \beta)$


\end{enumerate}

\newpage


\subsubsection{Truth Assignments}
\label{sec:Truth Assignments}
What does it mean for one wff to \textit{follow logically from other wffs}	?
When an expression $\alpha$ follows logically from another expression $\beta$ then no matter how the sentence symbols in $\beta$ and $\alpha$ are tanslated, if $\beta$ is true $\alpha$ will be true as well.\\
Example: $A_{1} $ follows logically from $A_{1}\land A_{2}$
\\That means: $(A_{1}\land A_{2})\rightarrow A_{1}$ is true.
\\Clearer Definition of the translation of sentence symbols:\\\\ \textbf{Truth Assignments}	\\
Sei $S$ a set of sentence Symbols.\\
Sei a set $\{F, T\}$ of truth values.\\
sei a function $v: S\rightarrow \{F,T\}$ the \textit{truth assignment}	 which determines whether each sentence symbol is true or false.\\
\\
Sei $\overline{S}$ the set of all wffs that can be built up from $S$ using the \hyperref[sec:well formed formulas]{formula-building operations}.\\

sei another function $\overline{v}: \overline{S}\rightarrow \{F, T\}$ for which 6 conditions hold:\\

Conditions for $\overline{v}: \overline{S}\rightarrow \{F, T\}$:\\
\begin{itemize}
	
	0.  	\forall A\in S(\overline{v}(A) = v(A))$\\
	\textit{this means $\overline{v}$ is an \hyperref[sec: extended/restricted functions]{extension} of $v$}	\\
\end{itemize}
\begin{enumerate}
	\item 	(
	$\overline{v}((\neg \alpha)) =$ 
	\begin{cases}
		T if $\overline{v}(\alpha)= F$	\\
		F\ otherwise
	\end{cases}$
	\item 
	\item 
	$\overline{v}((\alpha\land \beta))=$ \begin{cases}
	T\ if\ $\overline{v}(\alpha)= T \ and \ \overline{v}(\beta)=T $\\
	F \ otherwise
		\end{cases}
	\item  
	$\overline{v}((\alpha\lor \beta))=$ \begin{cases}
	T\ if\ $\overline{v}(\alpha)= T \ or \ \overline{v}(\beta)=T \ (or \ both) $\\
	F \ otherwise
		\end{cases}
	\item 
	$\overline{v}((\alpha \rightarrow  \beta))=$ \begin{cases}
	F\ if\ $\overline{v}(\alpha)= T \ and \ \overline{v}(\beta)=F $\\
	T \ otherwise
		\end{cases}
	\item $\overline{v}((\alpha \leftrightarrow  \beta))=$ \begin{cases}
	T\ if\ $\overline{v}(\alpha)= \overline{v}(\beta)$\\
	F \ otherwise
		
\end{enumerate}

	 










\newpage
\subsection{Induction}
\label{sec:Induction}
\subsubsection{the induction principle}
\label{sec:the induction principle}

\subsubsection{strong induction}
\label{sec:strong induction}

\subsubsection{the well ordering principle}
\label{sec:the well ordering principle}
\textbf{Every nonempty set of natural numbers has a smallest element.}
\\

\begin{math}
	\forall S \subseteq \mathbb{N}(S \neq \emptyset \rightarrow S
\end{math}
 has a smallest element)
 \\\\
 \textit{Proof.}	Suppose $S \subset \mathbb{N}$ and S does not have a smallest element.\\
 We will prove that $\forall n\in \mathbb{N}(n \notin S)$ meaning that $ S=\emptyset$\\\\
 We will prove this using strong induction.\\
 Suppose that $n\in\mathbb{N}$ and $\forall k<n(k \notin S)$\\
 Goal: $n\notin S$
 \\
 If $n\in S$ then n would be the smalles element of S since all elements smaller then n are not in S by the induction hypothesis. This is a contradiction to the assumption that S does not have a smalles element.
 \newpage
 \section{Relations}
 \label{sec:Relations}
 \subsection{functions}
 \label{sec:functions}
 
\subsubsection{extended/restricted functions}
\label{sec: extended/restricted functions}
A restriction of a function $f$ is a new function $f|_{A}$ which is the same function as $f$ its Domain is just a subset of the Domain of f:\\
$Dom(f)\subseteq Dom(f|_{A})$\\\\
Sei $f: A \rightarrow B$\\
Sei $C\subseteq A$\\
Sei $f|_{C}: C\rightarrow B$, wobei $f|_{C}\subseteq f$
\\$\rightarrow f|_{C}$ is a restriction of $f$ and $f$ is an extension of $f|_{C}$\\\\
In an extension of a function the domain is extended, in an restriction of a function the domain is restricted.

\newpage
\section{Linear algebra}
\label{sec:linear algebra}

\newpage

\section{Kombinatorik}
\label{sec:Kombinatorik}
\subsection{Permutationen}
\label{sec:Permutationen}
\subsubsection{Permutationen ohne Wiederholungen}
\label{sec:Permutationen ohne Wiederholungen}

Sei X eine Menge\\
\[	
	P_{x} = \{(a_{1}, ...., a_{|x|}): \{a_{1}, ..., a_{|x|} \}= X\}
\]
Die Permutationen $P_{x}$ von X sind also alle Möglichen unterschiedlichen Wege, die Elemente in X anzuordnen

\\Systematische Notierung der Permutationen von $P_{x+1}$:

Es gibt genau x+1 Stellen, an denen das "neue" Element $a_{x+1}$ stehen kann. Für jede dieser Konfigurationen gibt es für die anderen stellen $P_{x}$ Möglichkeiten die anderen Elemente anzuordnen.

$P_{3}$:
\\(1,2,\textbf{3}	)
\\(2,1,\textbf{3})

\\(1,\textbf{3},2)
\\(2,\textbf{3},1)

\\(\textbf{3},1,2)
\\(\textbf{3},2,1)
\\\\$P_{4}$:\\
(1,2,3,4)\ (1,2,4,3)\ (1,4,2,3)\ (4,1,2,3)\\
(2,1,3,4)\ (2,1,4,3)\ (1,4,2,3)\ (4,1,2,3)\\
(1,3,2,4)\ (1,3,4,2)\ (1,4,2,3)\ (4,1,2,3)\\
(2,3,1,4)\ (2,3,4,1)\ (1,4,2,3)\ (4,1,2,3)\\
(3,1,2,4)\ (3,1,4,2)\ (1,4,2,3)\ (4,1,2,3)\\
(3,2,1,4)\ (3,2,4,1)\ (1,4,2,3)\ (4,1,2,3)\\

Demnach gilt:
\[	
	P_{x+1} = (x+1)\cdot P_{x}
\]
Da $P_{0} = 1$, gilt $P_{n} = n! $

\subsubsection{Permutationen mit Wiederholungen}
\label{sec:Permutationen mit Wiederholungen}
Sei ein Tuple (1, 1, 2, 3).\\
Auf wie viele verschiedene Weisen kann dies angeordnet werden?\\
4! kann es nicht sein, da wenn die einsen vertauscht werden das selbe Tuple dabei herauskommt.
\\
 
Sei die beiden einsen seien unterschiedliche Objekte $1_{a}$ und $1_{b}$\\
Die berechnung der Variation ist nun eine Permutation ohne Wiederholung und kann durch 4! berechnet werden. \\

Für jede Variation in der ursprünglichen Variante kommen hier noch die Variationen innerhalb derselben Objekte hinzu, $P_{4}$ muss also durch die Anzahl der Möglichkeiten der Variationen innerhalb der gleichen elemente ($P_{2}$) geteilt werden.\\
$P_{(1,1,2,3)}=\frac{P_{4}}{P_{2}}	$




\end{document}
