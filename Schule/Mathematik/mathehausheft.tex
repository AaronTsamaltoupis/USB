\documentclass[12pt, letterpaper]{article}
\usepackage{hyperref}
\usepackage[utf8]{inputenc}
\usepackage{amsfonts}
\usepackage{amsmath}
\usepackage{lwarp-titlesec}
\usepackage{graphicx}
\graphicspath{\media\Aaron\USBStick}
\hypersetup{
    colorlinks=true,
    linkcolor=blue,
    filecolor=magenta,      
    urlcolor=cyan,
    pdftitle={Overleaf Example},
    pdfpagemode=FullScreen,
    }

\title{mathe hausheft}
\author{Aaron Tsamaltoupis}


\begin{document}
Aufgabe: 
Sei ein Würfel:
\begin{itemize}
	\item 1 feld der zahl 1
	\item 2 felder der zahl 2
	\item 3 felder der zahl 3
\end{itemize}
Sei dieser Würfel wird drei mal geworfen

\\Wie hoch ist die wahrscheinlichkeit, dass die summe der drei würfe größer als 6 ist?\\


Sei die Gesamtmenge $\Omega$ ist die Menge aller Möglichen dreierkombos bei drei Würfen\\
Berechnung von $\Omega$:\\
\[	
	12\cdot 11\cdot 10

\] 
sei nun eine liste aller möglichen würfe, die größer als 6 sind:
\begin{itemize}
	
	\item  sei 1 das erste elementt: 
die zweiten beiden elemente müssen zusammen größer gleich 6 sein.
Die einzige möglichkeit sind also zwei 3en, sowohl für den ersten, als auch den zweiten wurf gibt es für die 3en drei möglichkeiten\\\\
1 erstes element: $1\cdot 3\cdot 3$  würfe  = 9 würfe
\item sei 2 das erste element: \\
	die beiden folgenden elemente müssen größer gleich 4 sein.
	möglichkeiten: \begin{itemize}
		\item 2 x 2 ($2\cdot 2$ möglichkeiten)
		\item 1x2, 1x 3 ($2\cdot 3$möglichkeiten)
		\item 1x3, 1x2   ($2\cdot 3$möglichkeiten)
		plus alle möglichkeiten von vorher (9)

	\end{itemize}

formel für allle möglichen würfe, die größer als 6:\\
$3\cdot (möglichkeiten für würfe mit zwei würfeln größer gleich 6) + `$

\end{itemize}


\end{document}
