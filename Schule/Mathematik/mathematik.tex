\documentclass[12pt, letterpaper]{article}
\usepackage{hyperref}
\usepackage[utf8]{inputenc}
\usepackage{lwarp-titlesec}
\usepackage{graphicx}
\graphicspath{\media\Aaron\USBStick}
\hypersetup{
    colorlinks=true,
    linkcolor=blue,
    filecolor=magenta,      
    urlcolor=cyan,
    pdftitle={Overleaf Example},
    pdfpagemode=FullScreen,
    }

\title{MatheGK, Rausch}
\author{Aaron Tsamaltoupis}


\begin{document}
\maketitle
\tableofcontents
\newpage

\section{Q1: Calculus}
\label{sec:Calculus}
\newpage
\newpage
\section{Q2: Lineare Algebra}
\label{sec:Q2: Lineare Algebra}
\newpage
\section{Q3: Stochastik}
\subsection{Kombinatorik}
\label{sec:Kombinatorik}

\subsection{Stochastische Unabhängigkeit}
\label{sec:Stochastische Unabhängigkeit}
\begin{math}
\end{math}\\\\
Zwei Ereignisse A und B sind \textit{stochastisch unabhängig voneinander} iff \\
\begin{center}
	
\\\begin{math}
P_{B}(A) = P(A) \land  P_{A}(B) = P(B)$
\end{math}\\\\
\end{center}
Dabei gilt:\\
\begin{center}
	
\label{sec:Stochastik}
\[_{sei P(B) > 0\land P(A) > 0}\]
\[	
P_{A}(B)= \frac{P(A\cap B)}{P(A)}= \frac{|A\cap B|}{|A|}    	
\],\\
\[	
P_{B}(A)= \frac{P(B\cap A)}{P(B)}= \frac{|B\cap A|}{|B|}	\\\\\\
\],\\
\end{center}
\subsubsection{stochastisch abhängige Ereignisse}
\textit{Ein Ereignis B ist abhängig von einem anderen Ereignis A, wenn sich die Wahrscheinlichkeit von B ändert, wenn das Ereignis A eintritt.}	

	
\subsubsection{stochastisch unabhängige Ereignisse}


\newpage
\subsection{Vierfeldertafel}}
\label{sec:{Vierfeldertafel}}


\begin{tabular}{c|c|c|c}
    &$A&\overline{A}&Summe&
    \hline
    
    B&$|A\cap B|$&$|\overline{A}\cap B|$&$|B|$&
    \hline
    
    \overline{B}&$|\overline{B}\capA|$&$|\overline{B}\cap \overline{A}|$&$|\overline{B}|$& \hline Summe& $|A|$&$|\overline{A}|$& \Omega   
\end{tabular}
\\\\
Beispiel: Oktoberfest\\

T=Tourist\\
$\overline{T}$=Münchner\\
L = Lederhose\\
$\overline{L}$ = keine Lederhose\\\\
	\begin{tabular}{c|c|c|c|c}
		 &$L$&$\overline{L}$&Summe&Beschreibung&
		T&140&60&200&	\textit{Anzahl Touristen}&	
		\overline{T}&10&40&50&\textit{Anzahl Münchner}&
			    Summe&150&100&250$
	\end{tabular}
	\\\\
	Nach der \hyperref[sec:Stochastische Unabhängigkeit]{Formel bei stochastischer Abhängigkeit} gilt:
	\[
		P_{L}(T) = \frac{L\cap T}{L}=\frac{ 140}{150}\approx 93,33%	
	\]
\newpage
\subsection{Zufallsgröße}
\label{sec:Zufallsgröße}
\begin{tabular}
  g&g&g\\
\end{tabular}

\subsubsection{Standartabweichung}
\label{sec:Standartabweichung}
\begin{math}
  \sigma (X) = \sqrt{x_{1}-\mu)^{2}\cdot P(X-x_{1})+(x_{2}-\mu)^{2} \cdot P(X=x_{2}) +...+(x_{m}-\mu)^{2}\cdot P(X=x_{m}) }
\end{math}


\end{document}
