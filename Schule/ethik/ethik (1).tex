\documentclass[12pt, letterpaper]{article}
\usepackage{hyperref}
\usepackage[utf8]{inputenc}
\usepackage{lwarp-titlesec}
\usepackage{graphicx}
\graphicspath{\media\Aaron\USBStick}
\hypersetup{
    colorlinks=true,
    linkcolor=blue,
    filecolor=magenta,      
    urlcolor=cyan,
    pdftitle={Overleaf Example},
    pdfpagemode=FullScreen,
    }

    \title{Ethik}
\author{Aaron Tsamaltoupis}


\begin{document}
\maketitle
\tableofcontents
\newpage

\newpage
\section{Menschenrechte}
\label{sec:Menschenrechte}
\begin{itemize}
	\item unversell:
		\\jeder kann sich darauf berufen, sie gelten für alle Menschen
	\item egalitär\\
		-gelten für alle auf die gleiche Weise
	\item kategorisch, unbedingt:\\
		-benötigen keine Vorleistungen
	\item individuell, subjektiv\\
		nur der einzelne Mensch, das Individuum hat Menschenrechte
	\item sollten der Idee nach auch in jedem Rechtssystem juristisch einklagbar sein\\
		die eingliederung der menschenrechte in ein Rechtssystem bildet dann die Grundrechte dieses Systems
\end{itemize}
\subsection{Menschenwürde und Menschenrechte}
\label{sec:Menschenwürde und Menschenrechte}
\textbf{Beispiel Zwergenwerfen}	\\
\begin{itemize}
	\item Verletzung der Menschenwürde:\\
		Kleinwüchsige seien Objekte, die weggeworfen werden könnten
	\item Verletzung der freien Berufsausübung
\end{itemize}

\newpage
\section{Kant}
\subsection{Kants Menschenbild}
\label{sec:Kants Menschenbild}
\textbf{Der Mensch als Doppelwesen}	
\\\\
\begin{tabular}{ c| c }
	innere Welt & äußere Welt\\
	innere Verstandeswelt& äußere Sinneswelt\\
	\hline
	tierisches Geschöpf& Persönlichkeit (Intelligenz/Verstand)\\
	\hline
	bestimmt durch Lust/Unlust\\, Naturgesetze, fremdbestimmung& selbstbestimmt durch Vernunft, Sittengesetz
\end{tabular}
\\\\
Die moralische bewertung einer Handlung kann nicht objektiv wissenschaftlich bestimmt werden.\\
Das Wahrnehmbare der "äußeren Welt" reicht nicht  aus, um die Ethik zubegründen. 
	

\subsection{der Kategorische Imperativ}
\label{sec:der Kategorische Imperativ}

\label{sec:Kant}
\subsection{Der Mensch als Zweck an sich selbst}
\label{sec:Der Mensch als Zweck an sich selbst}
\begin{itemize}
	\item der mensch hat keinen Preis, sondern Würde
	\item dadurch kann der Mensch nie von einem anderen Menschen als Mittel zu einem Zweck gebraucht werden, er muss immer auch selber als Zweck für sich selbst gebraucht werden	
	\item jeder Mensch muss also die Würde der anderen Menschen achten und achten, dass sie keinen Preis haben  
	\item "Instrumentalisierungsverbot"
	\item "Selbstzweckhaftigkeit" der Menschen als Grund für die Menschenwürde
	\item Würde ist keine Qualität, die Menschen unterschiedlich stark ausgeprägt haben, sondern Menschen als vernünftige, moralische Wesen haben Würde
	\item Würde ist damit auch nicht abhängig von der geistigen Leistung, da der Einzelfall nicht wichtig ist

		

\end{itemize}
\newpage
\section{Aristoteles' Gerechtigkeitsbegriff}
\label{sec:Aristoteles' Gerechtigkeitsbegriff}
\begin{tabular}{cc}
	austeilende Geechtigkeit&ausgleichende Gerechtigkeit\\
	\hline
	Güter des Gemeinwesens werden verteilt:\\
		Geld\\
		Anerkennung\\
		Ämter\\
		Werte\\
		...
		
	&Tat vs. daraus entstandenene Konsequenz	
\end{tabular}
\newpage
\section{Liberalismus Egatilarismus}
\label{sec:Liberalismus Egatilarismus}
\textbf{Liberalismus}\\
\texttt{Freiheit}	
\\Aufgabe des Staates:
\\-garantiert Gleichheit: "schlanker Staat"
\\-Regeln für Wirtschaft (aus Rahmenbedingungen)
\\-Existenzminimum
\\-Ungleichheit positiv (fungiert als Motor)
\\\\\textbf{Raws}	\\
liberaler Egalitarismus/egalitärer Liberalismus
\\\\\textbf{Egatilarismus}	
\\\texttt{Gleichheit}	
\subsection{Rowl}
\label{sec:Rowl}

\end{document}
