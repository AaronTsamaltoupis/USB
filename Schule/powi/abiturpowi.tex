\documentclass[12pt, letterpaper]{article}
\usepackage{hyperref}
\usepackage[utf8]{inputenc}
\usepackage{amsfonts}
\usepackage{amsmath}
\usepackage{lwarp-titlesec}
\usepackage{graphicx}
\graphicspath{\media\Aaron\USBStick}
\hypersetup{
    colorlinks=true,
    linkcolor=blue,
    filecolor=magenta,      
    urlcolor=cyan,
    pdftitle={Overleaf Example},
    pdfpagemode=FullScreen,
    }

\title{Powi Abitur}
\author{Aaron Tsamaltoupis}

   \newtheorem{theorem}{Theorem}[section]
   \newtheorem{corollary}{Corollary}[theorem]
   \newtheorem{lemma}[theorem]{Lemma}

\begin{document}
\maketitle
\tableofcontents
\newpage\\
\textbf{Q1}\\ 
\textbf{Q1.1}\\ 
\textbf{Grundkurs} :\\
\begin{enumerate}
  
  \item Grundrechte und Rechtstaatlichkeit\\S.13, S.14, 42 (Sicherheit vs. Freiheit), S 62 (politische Theorien im Grundgesetz)\\\\
  \item Parlament, Länderkammer, Bundesregierung, europäische Institutionen im Gesetzgebungsprozess\\ S.19 (Aufgabenverteilung zwiscen Bund und Ländern), S 35f (Gewaltenteilung), 
    S 48 (Verschränkung der Verfassungsorgane), S 51 (Verfassungsorgane und gewaltenverschränkung)
  \item Rolle des Bundesverfassungsgerichts, Gewaltenteilung\\
    S. 57 (gewaltenteilung EU)

\end{enumerate}
\\\\
\textbf{Leistungskurs} :
\begin{enumerate}
  
  \item \\Das politische Mehrebenensystem Vor dem Hintergrund politischer theorien zur Gewalteneinteilung und gewaltenverschränkung (Montesqieu, Locke)\\ 
\begin{enumerate}
  \item S. 55, John Locke, Gewaltenteilung
  \item S. 51, Menschenbild Montesqieu, Locke
\end{enumerate}
\end{enumerate}
\\\\
\textbf{Q1.1} 
\textbf{Grundkurs} :\\
\begin{enumerate}
  \item politische Parteien als Möglichkeit der Partizipation (Funktion von Parteien, Populismus)\\
    S. 127 (bpb Populismus), S 155 Begriff Populismus, S. 169 Populismus
  \item alternative Formen politischer Beteiligung und Entscheidungsformen (bspw Volksentscheid)\\
    S. 
\end{enumerate}

\textbf{Leistungskurs:} \\
\begin{enumerate}
  \item Modelle des Wählerverhaltens, Wahlforschung\\
    S 90. (pluralismus und Willensbildung), S. 106 Wahlen, Wahlforschun, Herausforderung der Parteiendemokratie,
    S. 108: Instrumente der Wahlforschung
  \item Veränderungen von Parteiensystem und Parteientypen, innerparteiliche Demokratie\\
    S78 (Herausforderungen der Parteiendemokratie), S81 (Parlament vs Plebiszit), S89 , S. 94 (Aufgabe der Parteen), S. 97: Übergang Volksparteien zu professionalsierten Wählerparteien\\
    S. 100 innerparteiliche Politik, S 102 (innere Ordnung von Parteien)\\
    S. 115 (wehrhafte Demokratie)


  \item Identitäre vs. Repräsentative Denmokratie\\
    S. 71 (Identitäts, Konkurrenztheorie), S72 (Indentität vs Konkurrenz)
  \item Demokratietheorien der Gegenwart(Pluralismustheorien, deliberative Demokratietheorien)\\
    S. 64, S. 76 Übersicht\\
    Pluralismus: Russeau, Tocqueville , S. 70(Demokratie und Pluralismus), S. 79 (pluralismus und willensbildung in der BRD)\\
    S 90. (pluralismus und Willensbildung)\\
    deliberative Demokratie: ???\\
\end{enumerate}
\textbf{Q1.4}\\ 
\textbf{Grundkurs:} 
\begin{enumerate}
  \item Aufgaben, Funktionen, Probleme klassischer politischer Massenmedien,
    S. 157 definition Massenmedien, Medien im demokratischen Prozess,
    S. 161, Medien im politischen Prozess, S158, medien als vermittler, S. 159 f (Massenkommunikation),
    S. 164 (Medien und Demokratie), S 166: Fehler im Kommunikationsablauf
  \item Chancen und Risiken neuer politischer Kommunikationsformen im Internet, bspw Filterblasen, Fake News, Sicherheitsrisiko, digitalie Infrastruktur
  \item Veränderungen  im Verhältnis von Massenmedien und politischen Akteuern (Personalisierung, Medienethik)
\end{enumerate}
\textbf{Leistungskurs:}
\begin{enumerate}
  \item Medien als Wirtschaftsunternehmen\\
    S. 181 
  \item Pluralisierung, Internationalisierung, Fragmentierung politischer Öffentlichkeit
\end{enumerate}
\newpage
\section{Q2}
\label{sec:Q2}
\subsection{Q2.1}
\label{sec:Q2.1}
\textbf{Grundkurs} 
\begin{enumerate}
  \item Beobachtung, Analyse und Prognose wirtschaftlicher Konjunktur in offenen Volkswirtschaften durch Wirtschaftsforschungsinstitute\\
    S. 182 Grundwissen, S. 183, S185: Konjunkturanalyse,Konjunkturpolitik
    \item Grundlagen der keynesianischen stabilisierungspolitischen Konzeption (Krisenanalyse, Bedeutung der effektiven Gesamtnachfrage, Rolle des Staates, Multiplikatoreffekt) (S. 225, S. 258: Angebotspolitik, keynesianische Nachfragesteuerung)
    \item Möglichkeiten und Varianten nachfrageorientierter Politik (Fiskalpolitik, Geldpolitik) S. 224: Geldpolitik
    
    \item Probleme sowie politische und ökonomische Kontroversität nachfrageorientierter Fiskalpolitik, insbesondere Inflation sowie Staatsverschuldung\\
\end{enumerate}
\textbf{Leistungskurs} 
\begin{enumerate}
  \item Erkläungsmodelle konjunktureller Schwankung (güterwirtschaftlich, monetär)\\
    S. 201 (Ursachen konjunktureller Schwankungen), S.202 (konjunkturelle Indikatoren), S. 203 (Der Wirtschaftskreislauf)
  \item Erfahrungen mit fiskalpolitischen Interventionen im historischen Vergleich\\ S. 221
\end{enumerate}
\subsection{Q2.2}
\label{sec:Q2.2}
\textbf{Grundkurs} 
\begin{enumerate}
  \item Bedeutung und Bestimmungsfaktoren von Wirtschaftswachstum \\
    S. 206 (nachhaltiges Wachstum)
  \item Grundlagen der neoklassischen Konzeption (Einflussfaktoren auf das Wirtschafts), wirtschaftspolitische Gestaltung von Angebotsbedingungen \\
    Neoklassik: Marginalprinzip (homo oeconomicus inhärente Stabilität privater Sektor)
  \item Wettbewerbsfähigkeit von Staaten und Regionen im europäischen Binnenmarkt
    `
\end{enumerate}
\textbf{Leistungskurs}
\begin{enumerate}
  \item Wettbewerb in unterschiedlichen Marktformen, wirtschaftliche Konzentrationsprozesse
  \item Wettbewerbspolitik der EU (S. 227)

  \item wettbewerbspoliti
  \item WEttbewerbsfähigkeit von Staaten und Regionen im europäischen Binnenmarkt (227)
\end{enumerate}

\subsection{Q2.4 Arbeitsmarkt und Tarifpolitik}
\textbf{Grundkurs} 
\begin{enumerate}
  \item  Entwicklung von Beschäftigung, Fachkräftemangel, Beschäftigungsstrukturen (S. 264, S. 272, S. 261: arbeitsmarktpolitische Instrumente)
  \item Tarifvertragsparteien, Tarifpolitik, Tarifautonomie (S. 259f, S.265, S. 269: Tarifpartner, Tarifautonomie, politik, S. 271 Ablauf Tarifkonflikt)
  \item Entwicklung der Einkommens- und Vermögensverteilung (S.273)
  \item konkurrierende Gerechtigkeitsbegriffe (Bedarfs- und  Leistungsgerechtigkeit, Chancengleichheit, Diskriminierungsprobleme)\\
  


\end{enumerate}
\textbf{Leistungskurs} 
\label{sec:Q2.4}


\subsection{Q2.5}
\label{sec:Q2.5}

\newpage
\section{Q3}
\label{sec:Q3}
\subsection{Q3.1}
\label{sec:Q3.1}
\textbf{Grundkurs} 
\begin{enumerate}
  \item Russland-Ukraine Krieg, differenzeierte Staatenwelt, unterschiedliche Konfliktarten \\
    S. 298, S. 288\\
    Terrorismus: 314


  \item Ziele, Strategien deutscher Außen- und Sicherheitspolitik zu Konfliktbearbeitung und -prävention\\
    S. 301 (Kriegstüchtigkeit von Deutschland), S. 304

  \item Möglichkeiten, Verfahren, Akteure kollektiver Konfliktbvearbeitung und Friedenssicherung im Rahmen internationaler Institutionen und Bündnisse\\
    (dokument Sicherheitspolitisches Seminar), S.309: wandel von mono- zu multipolarität, S.306: zunehmende wichtigkeit von intern. Bündnissen : S. 323 (UNO), S. 325 (Völkerrecht), S.327, intern. Schutz der Menschenrechte, S. 330 , S.333 Nato
\end{enumerate}
\textbf{Leistungskurs} 
\begin{enumerate}
  \item Theorien internationaler Politik hinsichtlich Aspekte Frieden, Sicherheit, Kriegsursachen (Realismus, Idealismus/Liberalismus, Institutionalismus)
  \item Wandel staatlicher Souveränität durch Verrechtlichung (internationales Strafrecht) (S. 289, "alte" ,"neue" Kriege)

\end{enumerate}

\subsection{Q3.2}
\label{sec:Q3.2}
\textbf{Grundkurs} 
\begin{enumerate}
  \item Überblick über Entgrenzung, Verflechtung von Nationalökonomien hinsichlich Außenhandel, Freihandelszonen, Binnenmärkten, Währungsräumen, Währungssystemen, Kapitalmärkten, Arbeit und damit verbundenen Chancen und Risiken (Chancen und risiken von Globalisierung)\\
    S. 339 (was ist globalisierung), S. 340: Aspekte der Glob., S 341, \\
    S 344: Ursachen Folgen von Globalisierung, 
  \item Staaten zwischen Wohlfahrtsstaat und Wettbewerbsstaat\\
    S359, S. 363 (protektionismus), 
\end{enumerate}
\begin{enumerate}
  \item Außenhandelstheorien: S. 354 (Smith, Ricardo)

  
\end{enumerate}


\newpage
\section{Q1}
\label{sec:Q1}
\subsection{Q1.1}
\label{sec:Q1.1}
\textbf{Verfassung und Verfassungswirklichkeit: Rechtsstaatlichkeit und Verfassungskonflikte} 


\subsubsection{Grundrechte, Grundgesetz, Rechtsstaatlichkeit}
\label{sec:Grundrechte, Grundgesetz, Rechtsstaatlichkeit}

\textbf{Grundrechte und Rechtsstaatlichkeit in der Verfassung (insb. Art. 1, Art. 20, Art. 79 GG)} 

\begin{itemize}
  \item Artikel 1
  \label[sec: Artikel 1]
  \item Artikel 20: Die Verfassung in Kurzform
  \begin{itemize}
    \item Rechtsstaatlichkeit:
      \begin{itemize}
        \item Staat, indem auch die Gesetzgebung an die Verfassung gebunden ist $\implies$ Willkür ist so nicht möglich
        \item Schutz des individuellen Rechts
        \item sorgt für Gleichheit
        
      \end{itemize}
    \item Sozialstaat:
      \begin{itemize}
        \item  sozialdemokratischer Ansatz soziale Risiken der kapitalistischen Marktwirtschaft abzuschwächen
        \item staatliche Intervention in einem kapitalistischen System um soziale Ungleichheit zu verringern und das Existenzminimum der Bürger zu sichern, bspw durch Arbietsrecht, Tarifautonomie, Sozialversicherungspolitik, Jugen-, Kinder-, Familien, Inklusions-, Geschlechterpolitik
      \end{itemize}
    \item Demokratie
      \begin{itemize}
        \item Gewalt geht vom Volk aus, ermöglicht Mitbestimmung und benötigt auch Mitbestimmung des Volkes (so erfordert Demokratie bspw. Wahlen)
      \end{itemize}
    \item Bundesstaat
      \begin{itemize}
        \item Bündnis aus Bundesländer
        \item Länder haben andere Aufgaben als der Bund
        \item Länder müssen Gesetzen zustimmen
        \item Trotz Wahrung der Eigenständigkeit werden bestimmte Befugnisse an den Bund abgegeben
      \end{itemize}
    \item Republik
      \begin{itemize}
        \item Staatsoberhaupt wird gewählt (Gegensatz zur Monarchie) $\implies$ Gemeinwohlpolitik
        \item \hyperref[sec:Gewaltenteilung]{Gewaltenteilung} in Exekutive, Legislative, Judikative



  \end{itemize}
\item Artikel 79


  \item Freiheit vs Sicherheit\\
    Freiheit gilt als Basis der Menschenwürde und muss nicht gerechtfertigt werden\\
    

\end{itemize}

\newpage
\subsubsection{Verfassungsorgane}
\label{sec:Verfassungsorgane}

\textbf{Parlament, Länderkammer, Bundesregierung, Europäische Institutionen im Gesetzgebungsprozess} \\\\
S.48: Verschränkung der Verfassungsorgane in der Gesetzgebung\\\\
\textbf{Die Bundesregierung}
\begin{itemize}
  \item Art. 62 GG: Besteht aus dem Bundeskanzler und den Bundesministern
  \item Ausführung von Gesetzen, Umsetzung politischer Maßnahmen
  \item Rolle der Regierung in der Gesetzgebung:\\
    -Gesetzesentwürfe für den Bundestag\\
    -beratende, Unterstützende Rolle
  \item Deutsche Bundesregierung beruht auf dem Kanzlerprinzip: Der Bundeskanzler bestimmt die Richtlinien der Politik, trägt die Verantwortung
  \item Aufgabe des Kanzlers: 
    \begin{itemize}
      \item schlägt Minister zur Ernennung und Entlassung vor, kann den Wechsel von Ministern veranlassen, dadurch sehr viel Macht
      \item Darf sich neue Ministerien ausdenken
      \item trägt Verantwortung für Politik
    \end{itemize}
  \item Jeder Minister leitet seinen Zuständigkeitsbereich eigenständig und trägt für diesen Bereich die Verantwortung
\end{itemize}

\newpage
\textbf{Der Bundestag}
\begin{itemize}
  \item Nicht Teil der Exekutive
  \item Vertretung des deutschen Volkes
  \item Einzige Institution auf Bundesebene, die direkt vom deutschen Volk gewählt wird
  \item Aufgaben:\\
    -Gesetzgebung\\
    -Wahl des Kanzlers\\
    -Entscheidung über Einsatz der Bundeswehr\\

  \item Der Gesetzgebungsprozess:\\
    - Gesetzesentwurf wird eingebracht:\\
    $\implies$ \begin{itemize}
      \item Lesungen in Gesamtheit der Abgeordneten
      \item detailierte fachliche Bearbeitung, auch durch Anh;eorung von Experten
      \item 2.Lesung: Gesetzesentwurf wird diskutiert
      \item 3.Lesung: Abgeordnete stimmen über Entwurf ab
      \item Bundesrat und Bundestag bewerten Gesetzesentwurf
    \end{itemize}

\end{itemize}
\textbf{Der Bundesrat} 
\\
Landesparlamente sind zuständig für die Gesetzgebung in den Bereichen der Länder (Bildung, Kultur, etc...)
\\\\
Nach Art 50: Bundesrat ist Möglichkeit für die Läder bei der Gesetzgebung und Verwaltung auf Bundes- und EU-Ebene mitzuwirken.\\\\
$\implies$ dies beinhaltet: 
\begin{itemize}
  \item Bundesrat kann gegen Gesetze Einspruch erheben. Dieser Einspruch muss dann durch eine Mehrheit im Bundestag überstimmt werden.\\
    Oft können Gesetze nur mit ausdrücklicher Zustimmung des Bundesrats erlassen werden.
  \item Initiativrecht (Art 76 GG)\\
    Bundesrat kann (genauso wie Bundesregierung) Gesetzesentwürfe entwickeln.\\
    Bundesregierung


\end{itemize}
\\\\
\textbf{Der Bundespräsident} 
\begin{itemize}
  \item Repräsentant aller Deutschen
    \begin{itemize}
      \item völkerrechtliche Vertretung
      \item Staatsbesuche
      \item Ernennen von Soldaten, Bundesrichter, Bundesbeamten
    \end{itemize}
  \item sollte parteiplitisch neutral sein
  \item keine aktive politische Teilhabe
  \item kann Probleme in die Öffentlichkeit heben
  \item Rolle in der Gesetzgebung:\\
    -Prüfung der Gesetze, Unterzeichnung der Gesetze
  \item Wahl des Bundespräsidenten: \\
    -Bundesversammlung\\
    $\implies$ Bundesversammlung besteht zur Hälfte aus Bundestagsabgeordneten und zur Hälfte aus Vertretern der Länder\\

\end{itemize}




\newpage
\textbf{Das politische Mehrebenensystem vor dem Hintergrund politischer Theorien zur Gewaltenteilung und Gewaltenverschränkung}
\label{sec:Gewaltenteilung}\\
\subsubsection{Gewaltenteilung, Gewaltenverschränkung in der BRD}
\label{sec:Gewaltenteilung, Gewaltenverschrä}
Gewaltenteilung in der BRD:\\
Legislative:\\
Judikative: \\
Exekutive:\\

\subsubsection{John Locke, Montesqieu}
\label{sec:John Locke, Montesqieu}
\textbf{Menschenbild Locke:} \\
-Menschen sind gemeisnchaftsbilden, frei gleich, unabhängig\\
-Menschen haben unveräußerliche Rechte\\
-Der Mensch als Individuum steht über allem\\\\

Wenn jemand seiner Freiheit beraubt wird, darf er der Obrigkeit Widerstand leisten.\\\\

\textbf{Herrschaftsbild Locke:} \\
-unveränderliche Gesetze\\
-Volkssouveränität\\
-Trennung von Staat und Kirche\\
-Machtmissbrauch auch in Demokratien möglich\\
$\implies$ John Locke als Ideengeber der modernen Staaten
\\

\textbf{Menschenbild Montesqieu} \\
-Vernunft führt zum Allgemeinwohl\\
-Mensch hat Loyalität den gegebenen Gesetzen gegenüber\\
-gute Gesetze ergeben sich durch Vernunft und gottgegebene Ordnung\\

\\\textbf{Herrschaftsbild Motesqieu} \\
-Monarchie muss durch eine Verfassung begrenzt werden, um nicht zur Tyrannei zu werden\\
-Auch Demokratien können zu Tyranneien werden, wenn sie nicht durch eine Verfassung kontrolliert wird\\
-Tyrannei:
\begin{itemize}
  \item Regierenden überschreiten die Macht, die ihnen durch das Gesetz gegeben wurde
  \item Regierenden missbrauchen die Macht, die ihnen durch das Gesetz gegeben wurde in einer Weise, die gegen das Gesetz verstößt
  \item 
\end{itemize}
-Zweck der Gesellschaft ist die Erhaltung des Eigentums\\

  
\end{renz}
\end{document}
