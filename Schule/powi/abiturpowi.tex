\documentclass[12pt, letterpaper]{article}
\usepackage{hyperref}
\usepackage[utf8]{inputenc}
\usepackage{amsfonts}
\usepackage{amsmath}
\usepackage{lwarp-titlesec}
\usepackage{graphicx}
\graphicspath{\media\Aaron\USBStick}
\hypersetup{
    colorlinks=true,
    linkcolor=blue,
    filecolor=magenta,      
    urlcolor=cyan,
    pdftitle={Overleaf Example},
    pdfpagemode=FullScreen,
    }

\title{Powi Abitur}
\author{Aaron Tsamaltoupis}

   \newtheorem{theorem}{Theorem}[section]
   \newtheorem{corollary}{Corollary}[theorem]
   \newtheorem{lemma}[theorem]{Lemma}

\begin{document}
\maketitle
\tableofcontents
\newpage
\newpage
\section{Q1}
\subsection{Q1.1}
\label{sec:Q1.1}

\label{sec:Q1}
\textbf{Grundkurs} :\\
\begin{enumerate}
  
  \item Grundrechte und Rechtstaatlichkeit\\S.13, S.14, 42 (Sicherheit vs. Freiheit), S 62 (politische Theorien im Grundgesetz)\\\\
  \item Parlament, Länderkammer, Bundesregierung, europäische Institutionen im Gesetzgebungsprozess\\ S.19 (Aufgabenverteilung zwiscen Bund und Ländern), S 35f (Gewaltenteilung), 
    S 48 (Verschränkung der Verfassungsorgane), S 51 (Verfassungsorgane und gewaltenverschränkung)
  \item Rolle des Bundesverfassungsgerichts, Gewaltenteilung\\
    S. 57 (gewaltenteilung EU)

\end{enumerate}
\\\\
\textbf{Leistungskurs} :
\begin{enumerate}
  
  \item \\Das politische Mehrebenensystem Vor dem Hintergrund politischer theorien zur Gewalteneinteilung und gewaltenverschränkung (Montesqieu, Locke)\\ 
\begin{enumerate}
  \item S. 55, John Locke, Gewaltenteilung
  \item S. 51, Menschenbild Montesqieu, Locke
\end{enumerate}
\end{enumerate}
\\\\
\subsection{Q1.2}
\label{sec:Q1.2}
\textbf{Grundkurs} :\\
\begin{enumerate}
  \item politische Parteien als Möglichkeit der Partizipation (Funktion von Parteien, Populismus)\\
    S. 127 (bpb Populismus), S 155 Begriff Populismus, S. 169 Populismus
  \item alternative Formen politischer Beteiligung und Entscheidungsformen (bspw Volksentscheid)\\
    S. 
\end{enumerate}

\textbf{Leistungskurs:} \\
\begin{enumerate}
  \item Modelle des Wählerverhaltens, Wahlforschung\\
    S 90. (pluralismus und Willensbildung), S. 106 Wahlen, Wahlforschun, Herausforderung der Parteiendemokratie,
    S. 108: Instrumente der Wahlforschung
  \item Veränderungen von Parteiensystem und Parteientypen, innerparteiliche Demokratie\\
    S78 (Herausforderungen der Parteiendemokratie), S81 (Parlament vs Plebiszit), S89 , S. 94 (Aufgabe der Parteen), S. 97: Übergang Volksparteien zu professionalsierten Wählerparteien\\
    S. 100 innerparteiliche Politik, S 102 (innere Ordnung von Parteien)\\
    S. 115 (wehrhafte Demokratie)


  \item Identitäre vs. Repräsentative Denmokratie\\
    S. 71 (Identitäts, Konkurrenztheorie), S72 (Indentität vs Konkurrenz)
  \item Demokratietheorien der Gegenwart(Pluralismustheorien, deliberative Demokratietheorien)\\
    S. 64, S. 76 Übersicht\\
    Pluralismus: Russeau, Tocqueville , S. 70(Demokratie und Pluralismus), S. 79 (pluralismus und willensbildung in der BRD)\\
    S 90. (pluralismus und Willensbildung)\\
    deliberative Demokratie: ???\\
\end{enumerate}
\subsection{Q1.4}
\label{sec:Q1.4}
\textbf{Grundkurs:} 
\begin{enumerate}
  \item Aufgaben, Funktionen, Probleme klassischer politischer Massenmedien,
    S. 157 definition Massenmedien, Medien im demokratischen Prozess,
    S. 161, Medien im politischen Prozess, S158, medien als vermittler, S. 159 f (Massenkommunikation),
    S. 164 (Medien und Demokratie), S 166: Fehler im Kommunikationsablauf
  \item Chancen und Risiken neuer politischer Kommunikationsformen im Internet, bspw Filterblasen, Fake News, Sicherheitsrisiko, digitalie Infrastruktur
  \item Veränderungen  im Verhältnis von Massenmedien und politischen Akteuern (Personalisierung, Medienethik)
\end{enumerate}
\textbf{Leistungskurs:}
\begin{enumerate}
  \item Medien als Wirtschaftsunternehmen\\
    S. 181 
  \item Pluralisierung, Internationalisierung, Fragmentierung politischer Öffentlichkeit
\end{enumerate}
\newpage
\section{Q2}
\label{sec:Q2}
\subsection{Q2.1}
\label{sec:Q2.1}
\textbf{Grundkurs} 
\begin{enumerate}
  \item Beobachtung, Analyse und Prognose wirtschaftlicher Konjunktur in offenen Volkswirtschaften durch Wirtschaftsforschungsinstitute\\
    S. 182 Grundwissen, S. 183, S185: Konjunkturanalyse,Konjunkturpolitik
    \item Grundlagen der keynesianischen stabilisierungspolitischen Konzeption (Krisenanalyse, Bedeutung der effektiven Gesamtnachfrage, Rolle des Staates, Multiplikatoreffekt) (S. 225, S. 258: Angebotspolitik, keynesianische Nachfragesteuerung)
    \item Möglichkeiten und Varianten nachfrageorientierter Politik (Fiskalpolitik, Geldpolitik) S. 224: Geldpolitik
    
    \item Probleme sowie politische und ökonomische Kontroversität nachfrageorientierter Fiskalpolitik, insbesondere Inflation sowie Staatsverschuldung\\
\end{enumerate}
\textbf{Leistungskurs} 
\begin{enumerate}
  \item Erkläungsmodelle konjunktureller Schwankung (güterwirtschaftlich, monetär)\\
    S. 201 (Ursachen konjunktureller Schwankungen), S.202 (konjunkturelle Indikatoren), S. 203 (Der Wirtschaftskreislauf)
  \item Erfahrungen mit fiskalpolitischen Interventionen im historischen Vergleich\\ S. 221
\end{enumerate}
\subsection{Q2.2}
\label{sec:Q2.2}
\textbf{Grundkurs} 
\begin{enumerate}
  \item Bedeutung und Bestimmungsfaktoren von Wirtschaftswachstum \\
    S. 206 (nachhaltiges Wachstum)
  \item Grundlagen der neoklassischen Konzeption (Einflussfaktoren auf das Wirtschafts), wirtschaftspolitische Gestaltung von Angebotsbedingungen \\
    Neoklassik: Marginalprinzip (homo oeconomicus inhärente Stabilität privater Sektor)
  \item Wettbewerbsfähigkeit von Staaten und Regionen im europäischen Binnenmarkt
    `
\end{enumerate}
\textbf{Leistungskurs}
\begin{enumerate}
  \item Wettbewerb in unterschiedlichen Marktformen, wirtschaftliche Konzentrationsprozesse
  \item Wettbewerbspolitik der EU (S. 227)

  \item wettbewerbspoliti
  \item WEttbewerbsfähigkeit von Staaten und Regionen im europäischen Binnenmarkt (227)
\end{enumerate}

\subsection{Q2.4 Arbeitsmarkt und Tarifpolitik}
\textbf{Grundkurs} 
\begin{enumerate}
  \item  Entwicklung von Beschäftigung, Fachkräftemangel, Beschäftigungsstrukturen (S. 264, S. 272, S. 261: arbeitsmarktpolitische Instrumente)
  \item Tarifvertragsparteien, Tarifpolitik, Tarifautonomie (S. 259f, S.265, S. 269: Tarifpartner, Tarifautonomie, politik, S. 271 Ablauf Tarifkonflikt)
  \item Entwicklung der Einkommens- und Vermögensverteilung (S.273)
  \item konkurrierende Gerechtigkeitsbegriffe (Bedarfs- und  Leistungsgerechtigkeit, Chancengleichheit, Diskriminierungsprobleme)\\
  


\end{enumerate}
\textbf{Leistungskurs} 
\label{sec:Q2.4}


\subsection{Q2.5}
\label{sec:Q2.5}

\newpage
\section{Q3}
\label{sec:Q3}
\subsection{Q3.1}
\label{sec:Q3.1}
\textbf{Grundkurs} 
\begin{enumerate}
  \item Russland-Ukraine Krieg, differenzeierte Staatenwelt, unterschiedliche Konfliktarten \\
    S. 298, S. 288\\
    Terrorismus: 314


  \item Ziele, Strategien deutscher Außen- und Sicherheitspolitik zu Konfliktbearbeitung und -prävention\\
    S. 301 (Kriegstüchtigkeit von Deutschland), S. 304

  \item Möglichkeiten, Verfahren, Akteure kollektiver Konfliktbvearbeitung und Friedenssicherung im Rahmen internationaler Institutionen und Bündnisse\\
    (dokument Sicherheitspolitisches Seminar), S.309: wandel von mono- zu multipolarität, S.306: zunehmende wichtigkeit von intern. Bündnissen : S. 323 (UNO), S. 325 (Völkerrecht), S.327, intern. Schutz der Menschenrechte, S. 330 , S.333 Nato
\end{enumerate}
\textbf{Leistungskurs} 
\begin{enumerate}
  \item Theorien internationaler Politik hinsichtlich Aspekte Frieden, Sicherheit, Kriegsursachen (Realismus, Idealismus/Liberalismus, Institutionalismus)
  \item Wandel staatlicher Souveränität durch Verrechtlichung (internationales Strafrecht) (S. 289, "alte" ,"neue" Kriege)

\end{enumerate}

\subsection{Q3.2}
\label{sec:Q3.2}
\textbf{Grundkurs} 
\begin{enumerate}
  \item Überblick über Entgrenzung, Verflechtung von Nationalökonomien hinsichlich Außenhandel, Freihandelszonen, Binnenmärkten, Währungsräumen, Währungssystemen, Kapitalmärkten, Arbeit und damit verbundenen Chancen und Risiken (Chancen und risiken von Globalisierung)\\
    S. 339 (was ist globalisierung), S. 340: Aspekte der Glob., S 341, \\
    S 344: Ursachen Folgen von Globalisierung, 
  \item Staaten zwischen Wohlfahrtsstaat und Wettbewerbsstaat\\
    S359, S. 363 (protektionismus), 
\end{enumerate}
\begin{enumerate}
  \item Außenhandelstheorien: S. 354 (Smith, Ricardo)

  
\end{enumerate}


\end{document}
