\documentclass[12pt, letterpaper]{article}
\usepackage{hyperref}
\usepackage[utf8]{inputenc}
\usepackage{amsfonts}
   \usepackage{amssymb}
\usepackage{amsmath}
   \usepackage{amsthm}
\usepackage{graphicx}
\graphicspath{\media\Aaron\USBStick}
\hypersetup{
    colorlinks=true,
    linkcolor=blue,
    filecolor=magenta,      
    urlcolor=cyan,
    pdftitle={Overleaf Example},
    pdfpagemode=FullScreen,
    }

\title{hallo}
\author{Aaron Tsamaltoupis}

   \newtheorem{theorem}{Theorem}[section]
   \newtheorem{corollary}{Corollary}[theorem]
   \newtheorem{lemma}[theorem]{Lemma}

\begin{document}
\textbf{äußeres Design des Klassenzimmers}\\ 
Das Außendesign des Klassenzimmers sollte in dem Stil altgriechischer Tempel gebaut sein, aber trotzdem zu dem brutalistischen Rest der Schule passen. Die Säulen und das generelle Design des Tempels sollte deshalb nicht zu detailliert, sondern eher einfach gehalten sein. Dafür sollte das Innengebäude des Tempels deshalb auch dem restlichen Stil der Schule sehr stark ähneln und der gesamte Tempel nicht aus etwa Marmor, sondern aus Beton gebaut sein.\\
\textbf{innere Design des Klassenzimmers}\\ 
Trotz dem limitierten Platz, den das gesamte Gebäude bietet, muss das tatsächliche Klassenzimmer noch durch eine Wand abgetrennt sein vom Haupteingang, um eine vernünftige Lernatmosphäre zu bieten, ähnlich wie im E-Pavillion.\\
\textbf{Möbiliar}\\
Die Stühle sollten einheitlich mit dem altgriechischen Thema elegant sein, aber gleichzeitig nicht zu teuer sein, da hier im Umfeld eines Klassenzimmers keine Risiken eingegangen werden sollten.
Die Stühle haben dafür eine hohe Lehne und erinnern in ihrer Form an klassische, hochwertigere Bürostühle, sind aber durch ihre Materialien aus gebogenen Stahlrohr und Plastik robust und günstig.\\
Die Tische sind einfache Module, bestehend aus Holzkästen, die jeweils and den gegenüberliegenden Seiten geöffnet sind. So können sie leicht für die verschiedenen Anforderungen des Raumes verschoben werden, bspw für Gruppenarbeiten. Gleichzeitig können sie so auch für die Aufbewahrung von Materialien dienen.
\end{document}
